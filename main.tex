% vim: spell spelllang=es:
\input{preamble.tex}

\title{IA Búsqueda local}
\author{%
    Aleix Boné\\
    Alex Herrero\\
    Moisés Balcells\\
}
\date{%
}

% MARIA TERESA ABAD: <mabad@cs.upc.edu>

\begin{document} 
\maketitle


% La documentación deberá incluir:
% La descripción/justificación de la implementación del estado.
% La descripción/justificación de los operadores que habéis elegido...
% La descripción/justificación de las estrategias para hallar la solución inicial.
% La descripción/justificación de las funciones heurísticas.
% Para cada experimento:    
% • Condiciones de cada experimento
% • Resultados del experimento
% • Qué esperabais y qué habéis obtenido    
% • Comparaciones
% • Comentarios adicionales que os parezcan adecuados.
% Comparación entre los resultados obtenidos con Hill Climbing y Simulated Annealing (no olvidéis explicar cómo habéis ajustado los parámetros para este último algoritmo).
% Respuestas razonadas a las preguntas del enunciado.

\section{Implementación del estado}

\section{Operadores que hemos elegido}

\section{Estrategias para hallar la solución inicial}

\section{Funciones heurísticas}

\section{Experimentos}

\subsection{Influencia de los operadores}

\subsection{Influencia de la solución inicial}

\subsection{Parámetros para el simulated annealing}

\subsection{Evolución del tiempo de ejecución para valores crecientes de los parámetros}

\subsubsection{Número de usuarios que piden ficheros}

\begin{table}
\centering
\caption{Resultados del experimento 4.1}
\label{tab:ex4u}
\begin{tabular}{lrr}
\toprule
{} &        time &           ttt \\
USERS &             &               \\
\midrule
100   &    0.439545 &  2.427290e+05 \\
200   &    1.782818 &  4.995559e+05 \\
300   &    5.045900 &  7.645313e+05 \\
400   &   10.736700 &  1.034363e+06 \\
500   &   12.388000 &  1.263917e+06 \\
600   &   26.544800 &  1.543647e+06 \\
700   &   34.734900 &  1.785220e+06 \\
800   &   57.239900 &  2.029594e+06 \\
900   &   90.870700 &  2.305947e+06 \\
1000  &  117.450800 &  2.561704e+06 \\
\bottomrule
\end{tabular}
\end{table}


\begin{figure}[H]
\centering
\includegraphics[width=0.8\textwidth]{include/plots/ex4_u_time_bplot.pdf}
\caption{Experimento 7}%
\label{fig:ex7time}
\end{figure}

\subsubsection{Número de servidores (Manteniendo el número de replicaciones)}

\begin{tabular}{lrr}
\toprule
{} &     media &       std \\
NSERV &           &           \\
\midrule
50    &  4.795824 &  2.562619 \\
100   &  3.693740 &  1.447718 \\
150   &  3.343780 &  1.184884 \\
200   &  2.875980 &  1.162004 \\
250   &  2.675720 &  0.991247 \\
300   &  2.281000 &  0.777080 \\
350   &  2.326820 &  0.899757 \\
400   &  1.843920 &  0.842095 \\
450   &  1.869560 &  0.764063 \\
500   &  1.710100 &  0.580861 \\
550   &  1.834040 &  0.760056 \\
600   &  1.497540 &  0.524096 \\
650   &  1.707580 &  0.588698 \\
700   &  1.216540 &  0.444400 \\
750   &  1.368400 &  0.532025 \\
800   &  1.322060 &  0.544356 \\
850   &  1.266100 &  0.434882 \\
900   &  1.236180 &  0.580456 \\
950   &  1.079920 &  0.439555 \\
1000  &  1.134480 &  0.433452 \\
\bottomrule
\end{tabular}


\subsection{Diferencia entre Tiempo total de transmisión y tiempo para hallar la solución (Hill Climbing)}

\subsection{Diferencia entre Tiempo total de transmisión y tiempo para hallar la solución (Simulated Annealing)}

\subsection{Influencia del número de replicaciones de los ficheros}

\begin{table}
\centering
\caption{Resultados del experimento 7}
\label{tab:ex7}
\begin{tabular}{lrr}
\toprule
{} &     time &       ttt \\
NREP &          &           \\
\midrule
5    &   4.4126 &  546171.0 \\
10   &   6.1816 &  346898.4 \\
15   &   9.7479 &  267323.3 \\
20   &  12.4391 &  218621.1 \\
25   &  20.0942 &  202086.0 \\
\bottomrule
\end{tabular}
\end{table}


\begin{figure}[H]
\centering
\includegraphics[width=0.8\textwidth]{include/plots/ex7_time_bplot.pdf}
\caption{Experimento 7}%
\label{fig:ex7time}
\end{figure}

\begin{figure}[H]
\centering
\includegraphics[width=0.8\textwidth]{include/plots/ex7_ttt_bplot.pdf}
\caption{Experimento 7}%
\label{fig:ex7ttt}
\end{figure}

\subsection{Conclusión de los experimentos}
    
    
    
\end{document}